\documentclass[]{book}
\usepackage{lmodern}
\usepackage{amssymb,amsmath}
\usepackage{ifxetex,ifluatex}
\usepackage{fixltx2e} % provides \textsubscript
\ifnum 0\ifxetex 1\fi\ifluatex 1\fi=0 % if pdftex
  \usepackage[T1]{fontenc}
  \usepackage[utf8]{inputenc}
\else % if luatex or xelatex
  \ifxetex
    \usepackage{mathspec}
  \else
    \usepackage{fontspec}
  \fi
  \defaultfontfeatures{Ligatures=TeX,Scale=MatchLowercase}
\fi
% use upquote if available, for straight quotes in verbatim environments
\IfFileExists{upquote.sty}{\usepackage{upquote}}{}
% use microtype if available
\IfFileExists{microtype.sty}{%
\usepackage{microtype}
\UseMicrotypeSet[protrusion]{basicmath} % disable protrusion for tt fonts
}{}
\usepackage{hyperref}
\hypersetup{unicode=true,
            pdftitle={PIE Lab Manual},
            pdfauthor={The PIE lab},
            pdfborder={0 0 0},
            breaklinks=true}
\urlstyle{same}  % don't use monospace font for urls
\usepackage{natbib}
\bibliographystyle{apalike}
\usepackage{longtable,booktabs}
\usepackage{graphicx,grffile}
\makeatletter
\def\maxwidth{\ifdim\Gin@nat@width>\linewidth\linewidth\else\Gin@nat@width\fi}
\def\maxheight{\ifdim\Gin@nat@height>\textheight\textheight\else\Gin@nat@height\fi}
\makeatother
% Scale images if necessary, so that they will not overflow the page
% margins by default, and it is still possible to overwrite the defaults
% using explicit options in \includegraphics[width, height, ...]{}
\setkeys{Gin}{width=\maxwidth,height=\maxheight,keepaspectratio}
\IfFileExists{parskip.sty}{%
\usepackage{parskip}
}{% else
\setlength{\parindent}{0pt}
\setlength{\parskip}{6pt plus 2pt minus 1pt}
}
\setlength{\emergencystretch}{3em}  % prevent overfull lines
\providecommand{\tightlist}{%
  \setlength{\itemsep}{0pt}\setlength{\parskip}{0pt}}
\setcounter{secnumdepth}{5}
% Redefines (sub)paragraphs to behave more like sections
\ifx\paragraph\undefined\else
\let\oldparagraph\paragraph
\renewcommand{\paragraph}[1]{\oldparagraph{#1}\mbox{}}
\fi
\ifx\subparagraph\undefined\else
\let\oldsubparagraph\subparagraph
\renewcommand{\subparagraph}[1]{\oldsubparagraph{#1}\mbox{}}
\fi

%%% Use protect on footnotes to avoid problems with footnotes in titles
\let\rmarkdownfootnote\footnote%
\def\footnote{\protect\rmarkdownfootnote}

%%% Change title format to be more compact
\usepackage{titling}

% Create subtitle command for use in maketitle
\providecommand{\subtitle}[1]{
  \posttitle{
    \begin{center}\large#1\end{center}
    }
}

\setlength{\droptitle}{-2em}

  \title{PIE Lab Manual}
    \pretitle{\vspace{\droptitle}\centering\huge}
  \posttitle{\par}
    \author{The PIE lab}
    \preauthor{\centering\large\emph}
  \postauthor{\par}
      \predate{\centering\large\emph}
  \postdate{\par}
    \date{2019-10-31}

\usepackage{booktabs}

\begin{document}
\maketitle

{
\setcounter{tocdepth}{1}
\tableofcontents
}
\hypertarget{section}{%
\chapter*{}\label{section}}
\addcontentsline{toc}{chapter}{}

\includegraphics{images/PIElab5.png}

\hypertarget{about-the-lab}{%
\chapter{About the lab}\label{about-the-lab}}

\hypertarget{location}{%
\section{Location}\label{location}}

PIE lab -- Straub 417 and 419

\hypertarget{people}{%
\section{People}\label{people}}

\hypertarget{pis}{%
\subsection{PIs}\label{pis}}

\emph{David M. Condon}

\begin{itemize}
\tightlist
\item
  Office: Straub 323
\item
  Email: \href{mailto:dcondon@uoregon.edu}{\nolinkurl{dcondon@uoregon.edu}}
\item
  Twitter: \href{https://twitter.com/DMCpersonality}{@DMCpersonality}
\end{itemize}

\emph{Sara J. Weston}

\begin{itemize}
\tightlist
\item
  Office: Straub 325
\item
  Email: \href{mailto:sweston2@uoregon.edu}{\nolinkurl{sweston2@uoregon.edu}}
\item
  Twitter: \href{https://twitter.com/saraweston09}{@saraweston09}
\end{itemize}

\hypertarget{values}{%
\section{Values}\label{values}}

\hypertarget{transparency}{%
\subsection{Transparency}\label{transparency}}

Science must be verifiable by others. For this reason, we strive to be transparent in our work. We share code, materials and, when possible, data. We will admit mistakes.

\hypertarget{curiosity}{%
\subsection{Curiosity}\label{curiosity}}

\hypertarget{expectations}{%
\section{Expectations}\label{expectations}}

\hypertarget{honesty}{%
\subsection{Honesty}\label{honesty}}

\hypertarget{respect}{%
\subsection{Respect}\label{respect}}

\hypertarget{network-drive}{%
\section{Network Drive}\label{network-drive}}

Files can be shared among lab members using the lab network drive.

\hypertarget{accessing-the-drive-on-a-mac}{%
\subsection{Accessing the drive on a Mac}\label{accessing-the-drive-on-a-mac}}

\begin{enumerate}
\def\labelenumi{\arabic{enumi}.}
\tightlist
\item
  Go to `Connect to Server'.
\item
  If you're off-campus, connect to the UO VPN.
\item
  Enter the network address: \texttt{smb://cas-fs2.uoregon.edu/Psychology/a/PIElab}. (It's worth adding this as a favorite network so you don't have to remember the address again.)
\item
  Enter your DuckID and your password when prompted. Note that you can only gain access if you have been approved for access.
\end{enumerate}

\hypertarget{accessing-the-drive-on-a-pc}{%
\subsection{Accessing the drive on a PC}\label{accessing-the-drive-on-a-pc}}

\begin{enumerate}
\def\labelenumi{\arabic{enumi}.}
\tightlist
\item
  Follow directions \href{https://casit.uoregon.edu/faq/mapping-file-shares-in-windows-10}{here}.
\end{enumerate}

\hypertarget{resources}{%
\chapter{Resources}\label{resources}}

\hypertarget{uo-resources}{%
\section{UO resources}\label{uo-resources}}

\hypertarget{virtual-private-network-vpn}{%
\subsection{Virtual Private Network (VPN)}\label{virtual-private-network-vpn}}

Instructions are \href{https://casit.uoregon.edu/faq/distance-work/cisco-vpn}{here}.

\hypertarget{r}{%
\section{R}\label{r}}

\hypertarget{learning-r}{%
\subsection{Learning R}\label{learning-r}}

\begin{itemize}
\tightlist
\item
  \href{http://learningstatisticswithr-bookdown.netlify.com}{Learning Statistics with R} by Danielle Navarro
\item
  \href{http://r4ds.had.co.nz}{R for Data Science} by Hadley Wickham
\end{itemize}

\hypertarget{reference-guides}{%
\subsection{Reference guides}\label{reference-guides}}

\begin{itemize}
\tightlist
\item
  \href{https://www.rstudio.com/resources/cheatsheets/}{Cheatsheets}
\end{itemize}

\hypertarget{writing}{%
\section{Writing}\label{writing}}

\begin{itemize}
\tightlist
\item
  \href{http://www.dansimons.com/resources/writing_tips.html}{Dan Simons's writing guide}
\item
  \href{http://www.ejwagenmakers.com/2009/TeachingTipsWriting.pdf}{E.J. Wagenmaker's guide}
\end{itemize}

\hypertarget{presentations}{%
\section{Presentations}\label{presentations}}

\begin{itemize}
\tightlist
\item
  \href{https://mfr.osf.io/render?url=https\%3A\%2F\%2Fosf.io\%2Fd8wm9\%2Fdownload}{Presentation by Rachael Meager}
\end{itemize}

\hypertarget{talapas-overview}{%
\chapter{Talapas Overview}\label{talapas-overview}}

Talapas is the high performance computing cluster managed by Research Advanced Computing Services (aka RACS) at the University of Oregon. Since it is possible (and maybe even best) to get started with Talapas without knowing too much about the capabilities of the system, we will simply point to the \href{https://hpcf.uoregon.edu/content/talapas}{main website} and the \href{https://hpcrcf.atlassian.net/wiki/spaces/TCP/overview}{Talapas Knowledge Base} for learning more.

In this document, the numbered sections below will deal with separate tasks that users in the lab might want to perform on Talapas. If you learn how to do something that is not documented below, please add it to the list, even if you are unsure that others will find it useful. Or, just point to the documentation available online.

\hypertarget{requesting-access}{%
\section{Requesting Access}\label{requesting-access}}

Go to this \href{https://hpcrcf.atlassian.net/servicedesk/customer/portal/1}{website} to request access to the PIE lab Talapas account. Fill out the New Account Request Form, including \textbf{pielab} as the PIRG.

\hypertarget{using-open-ondemand-interactive-access}{%
\section{Using Open OnDemand Interactive Access}\label{using-open-ondemand-interactive-access}}

The RACS team recommends using an incognito tab in Chrome or Firefox before logging in to Talapas (Safari will not work), as this may help to limit the potential for someone to misuse your credentials. Bear in mind that computing services can be costly so there is more at risk than your personal info (and data).

In the browser window, go to:
\url{https://talapas-ln1.uoregon.edu}

Enter your username and password. If you don't know your credentials (or if you're not sure you have credentials), visit \href{https://hpcf.uoregon.edu/content/request-access}{this page} to request access. This will require the approval of a lab PI (Condon or Weston). Since there are fees associated with using the computing cluster, your request will be evaluated based on need and the availability of other options.

There is a brief overview of the OnDemand service worth reading --- see \href{https://hpcrcf.atlassian.net/wiki/spaces/TCP/pages/922746881/Open+OnDemand}{here}.

\hypertarget{loading-files-for-interactive-access}{%
\subsection{Loading files for Interactive Access}\label{loading-files-for-interactive-access}}

Once you have logged in, it's fairly intuitive to navigate the the dropdown menus at the top of the browser window. The `Files' dropdown gives several options and each will open a new window. For shared projects, consider using the `/projects/pielab/shared' directory to store your files; otherwise use the directory associated with your username \emph{within the pielab directory} (in other words, this one: `/projects/pielab/{[}username{]}'). Within the correct folder, you will want to create a new folder that is specific to each project as this is where you'll store the input and output files, just as you would if working on your own hard drive.

\hypertarget{to-launch-an-interactive-session}{%
\subsection{To launch an interactive session}\label{to-launch-an-interactive-session}}

From the OnDemand landing page, click `Interactive Apps'/`Talapas Desktop'. This will load a form that requires entry of the PIRG name (pielab) and a few other fields. There are several things to keep in mind when filling out the form to launch the virtual desktop on Talapas. First, the `partition' field requires selection of one of several options. For a complete list, with associated specs, limitations, and restrictions, see \href{https://hpcrcf.atlassian.net/wiki/spaces/TCP/pages/7285967/Partition+List}{here}. For many (most?) jobs, the default `interactive' partition should be fine. For more intense jobs, consider using `short' (gives up to 24 hours) or `long' (up to 2 weeks). Note that `interactive' has a max of 4 CPUs, which equates to about 16G RAM. For reference, my souped up laptop has 32G and a standard MacBook Pro has 16G or maybe 8G if older. Analyses of SAPA data from 2017 or later will routinely kill RStudio with only 16G of RAM (necessitating use of the `short' or `long' partitions).

For the number of hours, a good practice is to use the lesser of the max amount of time (this depends on the partition) or a generous guess at the length of time needed. If the analysis runs for a long while and the times out, you will have no output and will have spent the funds anyway. Better to have a chance of getting what you need than running out of time. However, you should also get into the practice of deleting instances that are no longer in use, as this will save money! Otherwise, the expenses will add up until the full time allotment has expired. For the number of cores and total memory, the standard seems to be 4G of memory for each CPU --- choose the CPUs accordingly (as best you can). Note however that you can also use GPUs, which seem to have more memory. See \href{https://hpcrcf.atlassian.net/wiki/spaces/TCP/pages/364773381/Memory}{here} for more info about guessing the right specifications.

For SAPA processing, I have done lots of experimenting. Unfortunately, most of the analyses conducted on these data are not able to be parallelized. This means the marginal benefit of invoking additional cores is often small (but not zero). When going from raw data to a version that is Dataverse-ready, this set-up seems to work pretty well: 0 GPU, 192 GB, 14 cores, on the short partition for at least 6 hours. If you seek to use more than 128 GB (as above), your request will end up being completed on either the GPU partition (which has 256 GB per node) or the fat partition (which has something even larger). But, for now anyway, the charges are based on the name of the partition specified. If you're in a rush, you could specify `short,gpu,fat' in the partition section of the form. This means it will run your job on the first available node, giving priority as listed. As of July 2019, the gpu partition costs twice as much as the short partition (\$0.008 per SU) and the fat partition costs 6x as much.

\hypertarget{to-determine-the-memory-usage-of-jobs-previously-run}{%
\subsection{To determine the memory usage of jobs previously run}\label{to-determine-the-memory-usage-of-jobs-previously-run}}

This has to be done through an ssh interface with Talapas. Open a shell and enter this at the prompt:

\begin{verbatim}
ssh username@talapas-ln1.uoregon.edu
\end{verbatim}

followed by your password.

To see the memory usage of recent jobs (since midnight of the current day), enter:

\begin{verbatim}
sacct --format='JobID,Elapsed,MaxRSS'
\end{verbatim}

To get more information on an individual job (recommended), enter:

\begin{verbatim}
seff <JobID>
\end{verbatim}

That last command will return something like:

\begin{verbatim}
$ seff 9609990

Job ID: 9609990
Cluster: mycluster
User/Group: username/talapas
State: CANCELLED (exit code 0)
Nodes: 1
Cores per node: 8
CPU Utilized: 00:23:25
CPU Efficiency: 3.10% of 12:36:16 core-walltime
Job Wall-clock time: 01:34:32
Memory Utilized: 29.12 GB
Memory Efficiency: 91.00% of 32.00 GB
\end{verbatim}

This job was cancelled after repeated efforts to compile. Note the high ratio of Memory Utilized to Memory Available. It is also possible to learn much more. Try using the code below by entering your own `username' and a date that goes back a few days in place of `year-mo-da'.

\begin{verbatim}
sacct -u <username> -S <year-mo-da> --format='JobID,Partition,State,AllocCPUS,ReqMem,MaxRSS,TimeLimit,Elapsed'
\end{verbatim}

This gives all of the most relevant info (in my opinion) on jobs from the date entered until now, including the final status of the job.

Finally, it is also possible to obtain information about specific files. This is useful if trying to determine how much time was needed to generate an output file. From the terminal, try a command like this:

\begin{verbatim}
ls -l ~/projects/<filename>
\end{verbatim}

This will give the size of the file along with the date and time of creation.

\hypertarget{to-run-a-live-session-of-rstudio}{%
\subsection{To run a live session of RStudio}\label{to-run-a-live-session-of-rstudio}}

Once you have begun the session (i.e., submitted the form), a new page will load. From here, click the blue button that says `Launch noVNC in New Tab'. This button may not be visibe right away as you sit in the queue, waiting for the necessary resources to become available. When you launch the tab, a virtual desktop will launch with a dropdown menu in the upper left called `Applications'. Choose `Education'/`Talapas RStudio'. Presumably you can figure out how to manage your workflow from here.

\hypertarget{ending-your-interactive-session}{%
\subsection{Ending your interactive session}\label{ending-your-interactive-session}}

Your analyses will continue running until completed, timed-out, or stopped for some other reason (coding errors, etc.). When your analyses are done, you should return to the Open OnDemand main page and click the red `Delete' button. This will stop your usage of the system and associated fees.

\hypertarget{using-talapas-for-batch-jobs}{%
\section{Using Talapas for Batch Jobs}\label{using-talapas-for-batch-jobs}}

Haven't done this yet! But see \href{https://hpcrcf.atlassian.net/wiki/spaces/TCP/pages/7286491/How-to+Submit+a+Job}{here} if you want to give it a go.

\hypertarget{onboarding}{%
\chapter{Onboarding}\label{onboarding}}

\hypertarget{citi-training}{%
\section{CITI training}\label{citi-training}}

All lab members must have appropriate CITI training in order to be included on IRB protocols and grant applications.

\hypertarget{where-to-get-citi-training}{%
\subsection{Where to get CITI training}\label{where-to-get-citi-training}}

Click on \href{https://www.citiprogram.org/Shibboleth.sso/Login?target=https\%3A\%2F\%2Fwww.citiprogram.org\%2FSecure\%2FWelcome.cfm\%3finst\%3d1831\&entityID=https\%3A\%2F\%2Fshibboleth.uoregon.edu\%2Fidp\%2Fshibboleth}{this link} in order to access the CITI Single Sign On (SSO) platform.

\emph{If you're a new user} log in using your DuckID and follow the instructions to create an account.

\hypertarget{required-courses}{%
\subsection{Required courses}\label{required-courses}}

\emph{Everyone}
- Protection of Human Research Subjects

\emph{For Grant Applications}
- Responsible Conduct of Research

\hypertarget{when-youre-done}{%
\subsection{When you're done}\label{when-youre-done}}

Save a copy of your certificates somwhere you can easily access them.

\hypertarget{website-and-lab-manual}{%
\chapter{Website and Lab Manual}\label{website-and-lab-manual}}

\hypertarget{website}{%
\section{Website}\label{website}}

\hypertarget{details}{%
\subsection{Details}\label{details}}

URL: \href{https://pielab-science.com}{pielab-science.com}

Username and password: email Sara or David for this information.

\hypertarget{purpose}{%
\subsection{Purpose}\label{purpose}}

PR for the PIE lab
- Who's part of the PIE lab
- What are we currently doing?
- (If actively recruiting) landing page for new recuirts
- (If ongoing study) landing page for study participants

\hypertarget{current-sections}{%
\subsection{Current sections}\label{current-sections}}

\begin{itemize}
\tightlist
\item
  \href{https://pielab-science.com/research}{What we study}
\item
  About
\item
  Current lab members (with a page for each)
\item
  Friends of the lab
\item
  \href{https://pielab-science.com/blog/}{Cooling on the rack} (news)
\end{itemize}

\hypertarget{how-to-update}{%
\subsection{How to update}\label{how-to-update}}

\begin{itemize}
\tightlist
\item
  Go to \href{https://pielab-science.com/admin}{pielab-science.com/admin} and log in (see above for username/password details)
\item
  See sections below for updating specific aspects
\end{itemize}

\hypertarget{your-profile}{%
\subsubsection{Your profile}\label{your-profile}}

\begin{itemize}
\tightlist
\item
  Go to the Media section (on the left-hand menu) and click Add New
\item
  Drop file into box, or select using menus
\item
  Go to Pages (left-hand menu) and hover over your page -- select Edit with Elementor
\item
  Use the interface to change text, upload a new picture, include links to other pages, whatever you want!
\item
  Be sure to click Update before leaving the page.
\end{itemize}

\hypertarget{your-cv}{%
\subsubsection{Your CV}\label{your-cv}}

\begin{itemize}
\tightlist
\item
  Go to the Media section (on the left-hand menu) and click Add New
\item
  Drop file into box, or select using menus
\item
  Go to Pages (left-hand menu) and hover over your page -- select Edit with Elementor
\item
  Hover over the box that contains the text. A pencil will show up in the corner; click on it.
\item
  On the left-hand side, an Edit Text Editor menu will show up. Find the link to your CV. If you click on it, the link to the PDF will appear. You can either
  - Edit the link to match the new upload (involves changing the year and month of upload and the name of the file as it was uploaded to Wordpress), or
  - Delete the current link to the CV and add a new one with the Add Media button.
\item
  Be sure to click Update before leaving the page.
\end{itemize}

\hypertarget{troubleshooting}{%
\subsubsection{Troubleshooting}\label{troubleshooting}}

\emph{If the page doesn't update}

\begin{itemize}
\tightlist
\item
  Wait 15 minutes and reload the webpage.
\item
  Clear the wordpress Cache

  \begin{itemize}
  \tightlist
  \item
    On the wordpress dashboard, click on Mangaged Wordpress (top) and select Flush Cache. This will take a minute or so.
  \end{itemize}
\item
  Clear your browser Cache

  \begin{itemize}
  \tightlist
  \item
    This will be under your browser's settings.
  \item
    Try opening the webpage in an incognito window as well, to see if this is a problem.
  \end{itemize}
\end{itemize}

\hypertarget{lab-manual}{%
\section{Lab Manual}\label{lab-manual}}

\hypertarget{clone-lab-manual-to-your-computer}{%
\subsection{Clone lab manual to your computer}\label{clone-lab-manual-to-your-computer}}

\emph{Note: this is only available to current graduate students and PIs in the lab.}

\textbf{Necessary materials:}

\begin{itemize}
\tightlist
\item
  Git
\item
  GitHub account with access to \href{https://github.com/orgs/pie-lab}{PIElab organization}
\item
  Rstudio that is connected to GitHub through your personal account.
\end{itemize}

\begin{enumerate}
\def\labelenumi{\arabic{enumi}.}
\tightlist
\item
  Go to \href{https://github.com/pie-lab/manual}{manual repository}.
\item
  Click on the green \emph{Clone or download} button.
\item
  Copy the link that appears.
\item
  Open RStudio. Create a New Project.
\item
  Select Version Control and then Git. Name the project ``manual'' (so it matches the Rproj file that already exists in the repository. Save this project somewhere you will remember.
\end{enumerate}

\hypertarget{updating-the-manual-using-the-rstudio-gui}{%
\subsection{Updating the manual {[}using the RStudio GUI{]}}\label{updating-the-manual-using-the-rstudio-gui}}

\textbf{Before you make ANY changes:}

\begin{enumerate}
\def\labelenumi{\arabic{enumi}.}
\tightlist
\item
  Open the manual RStudio project.
\item
  Pull the most recent version of the manual using the blue down arrow.
\end{enumerate}

\textbf{Make any changes to the manual as you think are necessary.}

\begin{enumerate}
\def\labelenumi{\arabic{enumi}.}
\setcounter{enumi}{2}
\tightlist
\item
  In the Build tab in Rstudio, click on \emph{Build Book}.
\item
  In the Git tab, click Commit. In the commit message section, briefly (4-5 words max) describe the changes you made. Example: wrote updating manual section. In the window on the left, the easiest thing to do is select everything and click \emph{Stage}. Ensure all the boxes are checked. This will be slower than only selecting the parts of the book that have changed. However, the nature of an RMarkdown book is many changes affect most parts of the book (because it changes you navigate to them). If you miss a change you have made, the book may not render properly online. Click the \emph{Commit Button}.
\item
  Click on the green up arrow to push the book to GitHub, which will update the book online.
\end{enumerate}

\hypertarget{updating-the-manual-using-the-terminal-in-rstudio}{%
\subsection{Updating the manual {[}using the terminal in Rstudio{]}}\label{updating-the-manual-using-the-terminal-in-rstudio}}

\textbf{Before you make ANY changes:}

\begin{enumerate}
\def\labelenumi{\arabic{enumi}.}
\tightlist
\item
  Open the manual RStudio project.
\item
  In the terminal window of RStudio, pull the most recent version using this code:
\end{enumerate}

\begin{verbatim}
git pull
\end{verbatim}

\textbf{Make any changes to the manual as you think are necessary.}

\begin{enumerate}
\def\labelenumi{\arabic{enumi}.}
\setcounter{enumi}{2}
\tightlist
\item
  In the Build tab in Rstudio, click on \emph{Build Book}.
\item
  In the Terminal, commit your changes. The easiest way to do this is to type:
\end{enumerate}

\begin{verbatim}
git add .
\end{verbatim}

where the period indicate to include anything. Alternatively, you can add only specific files, if you don't want to commit everything. For example:

\begin{verbatim}
git add file.R
\end{verbatim}

or:

\begin{verbatim}
git add path/file.R
\end{verbatim}

To commit your changes, type:

\begin{verbatim}
git commit -m "message here"
\end{verbatim}

Be sure to include a message describing the changes. Keep the message short.

Finally upload your changes with:

\begin{verbatim}
git push
\end{verbatim}

You're done!

\hypertarget{requesting-updates}{%
\subsection{Requesting updates}\label{requesting-updates}}

You may be unable to make changes to the lab manual because you are not authorized to make changes or don't know the answer to the question. In either of those cases, you make create an issue on the GitHub page to alert the lab to the need for additional material. To do so:

\begin{enumerate}
\def\labelenumi{\arabic{enumi}.}
\tightlist
\item
  Go to the \href{https://github.com/pie-lab/manual/issues}{Issues tab} on the GitHub page for the lab manual.
\item
  Click on New Issue and complete the form. Be sure to include as much detail as you can about the issue you have. The more detailed your request, the better the information in the manual will be.
\end{enumerate}

\emph{Note: create one issue request for each question you have. Don't put lots of requests into a single issue.}

Other lab members can take responsibility for this issue by self-assigning the issue. Once material has been added to the manual, the member responsible for resolving the issue should reply to the original poster, including a link to the relevant section. If this meets the need of the original poster, they should reply so and the issue can be marked as closed.

\bibliography{book.bib}


\end{document}
